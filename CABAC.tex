%!TeX program = xelatex
\documentclass{/home/hi/Study/template/code}
%--------------------------------- Head -------------------------------------
\title{CABAC}
\author{\textcopyright Yang-datong }
\date{2024-08-20 14:38}

\begin{document}
\makehead
%--------------------------------- Body -------------------------------------
\section{CABAC算法}
CABAC(Context-Adaptive Binary Arithmetic Coding)是一种用于视频压缩的熵编码算法,广泛应用于H.264/AVC和HEVC(H.265)视频编码标准中。CABAC通过利用上下文信息来自适应地编码二进制符号,从而实现高效的压缩。其主要特点:
\begin{itemize}
	\item 上下文自适应:CABAC根据符号的上下文信息(如前面符号的统计特性)来选择概率模型,从而提高编码效率。
	\item 二进制算术编码:CABAC使用算术编码对二进制符号进行编码,这种方法比传统的霍夫曼编码更接近熵极限,能更好地压缩数据。
	\item 高压缩率:由于其自适应性和高效的算术编码,CABAC通常能比其他熵编码方法(如CAVLC)提供更高的压缩率。
\end{itemize}

CABAC编码过程主要包括以下几个步骤:
\begin{itemize}
	\item 二进制化:将非二进制符号(如运动矢量、残差系数等)转换为二进制符号。
	\item 上下文建模:根据符号的上下文信息选择合适的概率模型。上下文信息可以包括前面符号的值、位置等。
	\item 算术编码:使用选定的概率模型对二进制符号进行算术编码。
\end{itemize}

\subsection{文本算术编码}

\subsection{二进制算术编码}

%--------------------------------- Reference --------------------------------
\newpage
\begin{thebibliography}{1}
	\bibitem{a} 作者. \emph{文献}[M]. 地点:出版社,年份.\url{www.google.com}
\end{thebibliography}

\end{document}
