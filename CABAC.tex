%!TeX program = xelatex
\documentclass{/Users/hi/Study/template/code}
%--------------------------------- Head -------------------------------------
\title{CABAC}
\author{\textcopyright Yang-datong }
\date{2024-08-20 14:38}

\begin{document}
\makehead
%我想让你扮演一名音视频编解码算法老师。我将提供一些音视频概念,你的工作是用易于理解的术语来解释它们。这可能包括提供解决问题的分步说明、用视觉演示各种技术或建议在线资源以供进一步研究。我的第一个请求是“文本算术编码中,如何对文本进行二进制化? 截断莱斯二元化是什么?K阶指数哥伦布二元化是什么?定长二元化是什么?”

%CABAC 仅在标准的主要和更高配置文件(而不是扩展配置文件)中受支持,因为与称为上下文自适应的更简单方案相比,它需要大量的处理来解码标准的基线配置文件中使用的可变长度编码(CAVLC)。 CABAC 也很难并行化和矢量化,因此可以结合使用其他形式的并行性(例如空间区域并行性)。在 HEVC 中,CABAC 用于该标准的所有配置文件中。

%--------------------------------- Body -------------------------------------
\section{算术编码}
\subsection{文本算术编码}
早期的压缩方法(如霍夫曼编码)在处理符号时存在一定的局限性,尤其是在符号频率分布不均的情况下。为了提高数据压缩的效率于是有了算术编码。

算术编码是一种无损数据压缩技术,它通过将整个消息视为一个单一的数字区间来实现压缩。与传统的霍夫曼编码不同,算术编码不使用离散的符号,而是将整个消息编码为一个在0到1之间的浮点数。

算术编码的基本步骤如下:
\begin{serialNumber}
	\item 概率模型:为每个符号分配一个概率,通常基于历史数据。
	\item 区间划分:根据符号的概率,将区间[0, 1)划分为多个子区间。
	\item 逐步缩小区间:对于每个符号,根据其概率区间逐步缩小当前区间,直到处理完所有符号。
	\item 输出编码:最终输出的编码是当前区间的一个浮点数,通常选择区间的中点。
\end{serialNumber}

算术编码分为浮点算术编码与定点算术编码,对浮点算术编码而言,用$[0, 1]$的概率区间,对一串字符编码后,得到了最终区间,在最终区间挑选一个数字作为编码输出,而这个数字是一个小数,受计算机精度的影响。
\begin{remark}
	为了避免浮点数的计算效率低的情况,在实际使用中采用定点算术编码,且根据计算机的精度采用比例缩放的方法来解决。在H264/H265中将[0, 1]区间放到至[0,210],采用32位的寄存器实现。
\end{remark}

\begin{tcolorbox}
	\small
	假设要编码字符串 "ABAC"。我们首先需要定义每个符号的概率。假设我们有以下概率分布:
	%\newcolumntype{L}{>{\arraybackslash}p{5cm}}
	\begin{longtable}{ccc}
		%------------ Name ---------------
		%------------ Head ----------------
		\toprule
		\textbf{A} & \textbf{B} & \textbf{C} \\
		\midrule
		\endfirsthead
		%---------- Breakable --------------
		\bottomrule()
		\multicolumn{3}{r}{续下页}
		\endfoot
		%------------ Bottom ---------------
		\bottomrule()
		\endlastfoot
		%------------- Main ----------------
		0.5        & 0.25       & 0.25
	\end{longtable}

	\begin{serialNumber}
		\item 定义区间:根据每个符号的概率,定义每个符号对应的区间:
		%\newcolumntype{L}{>{\arraybackslash}p{5cm}}
		\begin{longtable}{ccc}
			%------------ Name ---------------
			%------------ Head ----------------
			\toprule
			\textbf{A} & \textbf{B}  & \textbf{C}  \\
			\midrule
			\endfirsthead
			%---------- Breakable --------------
			\bottomrule()
			\multicolumn{3}{r}{续下页}
			\endfoot
			%------------ Bottom ---------------
			\bottomrule()
			\endlastfoot
			%------------- Main ----------------
			[0.0, 0.5) & [0.5, 0.75) & [0.75, 1.0) \\
		\end{longtable}

		\item 初始化区间:初始化一个区间 [low, high) 为 [0.0, 1.0)。
		\item 逐个符号编码:逐个处理字符串中的符号,并更新当前区间。
		\begin{equation}
			[low,high) \rightarrow  [low + (high - low)Symbol_{low} , low + ( high - low )Symbol_{high}]
		\end{equation}
		\begin{itemize}
			\item  处理第一个符号 'A': 当前区间: [0.0, 1.0),新区间计算:low = 0.0,high = 0.0 + (1.0 - 0.0)  0.5 = 0.5,更新后的区间: [0.0, 0.5)
			\item  处理第二个符号 'B': 当前区间: [0.0, 0.5) ,新区间计算: low = 0.0 + (0.5 - 0.0)  0.5 = 0.25 ,high = 0.0 + (0.5 - 0.0)  0.75 = 0.375 ,更新后的区间: [0.25, 0.375)
			\item  处理第三个符号 'A': 当前区间: [0.25, 0.375) ,新区间计算: low = 0.25 + (0.375 - 0.25)  0.5 = 0.3125 ,high = 0.25 + (0.375 - 0.25)  0.5 = 0.34375 ,更新后的区间: [0.3125, 0.34375)
			\item  处理第四个符号 'C': 当前区间: [0.3125, 0.34375) ,新区间计算: low = 0.3125 + (0.34375 - 0.3125)  0.75 = 0.340625 ,high = 0.3125 + (0.34375 - 0.3125)  1.0 = 0.34375 ,更新后的区间: [0.340625, 0.34375)
		\end{itemize}

		\item 输出编码值:选择区间内的任意一个值作为编码结果,通常选择区间的中点。这里可以选择:
		\begin{equation}
			\text{编码值} = \frac{0.340625 + 0.34375}{2} = 0.3421875
		\end{equation}
	\end{serialNumber}
\end{tcolorbox}


\subsection{二进制算术编码}
文本算术编码处理的是字符数据,通常是ASCII或Unicode字符,需要处理字符集和更复杂的概率模型。而二进制算术编码处理的是二进制数据(0和1),通常只有两个符号(0和1),概率分布简单。

\subsubsection{二进制化}
\paragraph{定长二元化}
定长二元化(Fixed-Length Coding)是一种简单的编码方法,其中每个符号都被分配一个固定长度的二进制代码。其特点包括:

固定长度:每个字符或符号的二进制表示长度相同,例如,使用8位表示所有字符。

简单性:由于每个符号的长度相同,解码过程非常简单。

效率:在某些情况下,定长编码可能会导致空间浪费,尤其是当字符集较小而编码长度较大时。

例如,在ASCII中,字符'A'的二进制表示是01000001。


\paragraph{截断莱斯(TU)二元化}
截断莱斯二元化(Truncated Rice Coding)是一种用于无损数据压缩的编码方法。它的基本思想是:
\begin{serialNumber}
	\item 分段编码:选择一个参数$k ( k > 0 , k \in N )$,由$k$决定编码的分段大小,从而将数据分成多个段,每个段的长度是可变的。
	\item 商和余数:将要编码的数$ N $表示为商和余数的形式:\( N = q \cdot 2^k + r \),其中\( q \)是商,\( r \)是余数。
	\item 编码商:用一串\( q \)个1后跟一个0来表示商。如,$q = 3$则编码为"1110"
	\item 编码余数:用\( k \)位二进制数来表示余数。如, 余数为2,但$k = 3$,则编码为"010"
	\item 截断:在编码过程中,如果某个段的长度超过预设的最大值,则将其截断,以避免过长的编码。
\end{serialNumber}
C++实现代码见文末。



\paragraph{K阶指数哥伦布(UEGK)二元化}
K阶指数哥伦布二元化(K-th Order Exponential Golomb Coding)是一种改进的哥伦布编码方法,也是一种变长编码方法,适用于无符号整数、具有特定概率分布的数据。其步骤包括:
%  TODO YangJing  <24-08-24 00:23:46> % 
%https://blog.51cto.com/u_15296426/3029428
%https://www.cnblogs.com/SoaringLee/p/10532499.html
%\begin{serialNumber}
%	\item 概率模型:根据数据的概率分布,选择合适的$k$值。
%	\item 编码过程:使用K阶指数函数来生成编码,通常用于表示较大的整数,比如要编码的数$N$表示为$( N + 2^{k} )$。
%	\item 输出二进制:将生成的编码转换为二进制形式。
%	\item 前缀和后缀:将二进制数分为前缀和后缀,前缀部分由一串0和一个1组成,后缀部分是剩余的二进制数。
%	\item 组合:将前缀和后缀组合起来,得到最终的编码。
%\end{serialNumber}

\subsubsection{算术编码}
由于输入只有两个符号:“0”,“1”,那么简单得根据实际情况'1'的个数多则MPS = 1, LPS = 0。反之,MPS = 0 , LPS = 1,根据符号的出现概率,将区间划分为不同的子区间。例如,MPS = 1 的概率为 0.8,LPS = 0 的概率为 0.2,那么可以将区间 [0, 1) 划分为 [0, 0.8) 和 [0.2, 1)。

\begin{tcolorbox}
	\small
	假设要编码的二进制序列是 110010,且已知 MPS = 1 (概率 0.8),LPS = 0 (概率 0.2)。
	\begin{serialNumber}
		\item 定义区间:根据每个符号的概率,定义每个符号对应的区间:
		%\newcolumntype{L}{>{\arraybackslash}p{5cm}}
		\begin{longtable}{cc}
			%------------ Name ---------------
			%------------ Head ----------------
			\toprule
			\textbf{MPS} & \textbf{LPS} \\
			\midrule
			\endfirsthead
			%---------- Breakable --------------
			\bottomrule()
			\multicolumn{2}{r}{续下页}
			\endfoot
			%------------ Bottom ---------------
			\bottomrule()
			\endlastfoot
			%------------- Main ----------------
			[0.0, 0.8)   & [0.8, 1.0)   \\
		\end{longtable}

		\item 初始化区间:初始化一个区间 [low, high) 为 [0.0, 1.0)。
		\item 逐个符号编码:逐个处理二进制序列中的符号,并更新当前区间。
		\begin{equation}
			\begin{cases}
				range = high - low                                                                    \\
				[low,high) \rightarrow  [low ,  low' + range * P( MPS ) )                 & For ~ MPS \\
				[low,high) \rightarrow  [low + range * P( MPS ) ,low' + range * P( LPS )) & For ~ LPS
			\end{cases}
		\end{equation}
		其中,low'指的是当前区间的低区间计算后的新低区间。

		\begin{itemize}
			\item 编码第一个符号 1:区间 [0, 1) 划分为 [0, 0.8) 和 [0.8, 1)。选择 MPS 区间 [0, 0.8)。
			\item 编码第二个符号 1:新区间 [0, 0.8) 划分为 [0, 0.64) (MPS) 和 [0.64, 0.16) (LPS)。选择 MPS 区间 [0, 0.64)。
			\item 编码第三个符号 0:新区间 [0, 0.64) 划分为 [0, 0.64 * 0.8) (MPS) 和 [0.64 * 0.8, 0.64 * 0.8 + 0.64 * 0.2) (LPS)。选择 LPS 区间 [0.512, 0.64)。
			\item 编码第四个符号 0:新区间 [0.512, 0.64) 划分为 [0.512, (0.64 - 0.512) * 0.8) (MPS) 和 [0.512 + (0.64 - 0.512) * 0.8 ,0.6144 + (0.64 - 0.512) * 0.2 ) (LPS)。选择 LPS 区间 [0.6144, 0.64)。
			\item 编码第五个符号 1:新区间 [0.6144, 0.64) 划分为 [0.6144, 0.6144 + (0.64-0.6144)*0.8) (MPS) 和 [0.6144 + (0.64-0.6144)*0.8, 0.63488 + (0.64-0.6144)*0.2) (LPS)。选择 MPS 区间 [0.6144, 0.2048),选择MPS区间为 [0.6144,0.63488)。
			\item 编码第六个符号 0:新区间 [0.6144, 0.63488) 划分为 [0.6144, 0.6144 + 0.02048 * 0.8) (MPS) 和 [0.6144 + 0.02048 * 0.8,0.630784 + 0.02048 * 0.2) (LPS)。选择 LPS 区间 [0.630784, 0.63488)。
		\end{itemize}

		\item 输出编码值:选择区间内的任意一个值作为编码结果,通常选择区间的中点。这里可以选择:
		\begin{equation}
			\text{编码值} = \frac{0.630784 + 0.63488}{2} = 0.632832
		\end{equation}
	\end{serialNumber}


\end{tcolorbox}

\begin{tcolorbox}
	\small
	对0.632832进行解码,已知条件为初始化区间[0,1),以及P(MPS) = 0.8 , P(LPS) = 0.2
	\begin{serialNumber}
		\item 计算范围: range = high - low = 1 - 0 = 1,
		判断第一个符号: mid = low + range * P(MPS) = 0 + 1 * 0.8 = 0.8,由于 value < mid,所以符号为 1 (MPS),
		更新区间:high = mid = 0.8

		\item 计算范围: range = high - low = 0.8 - 0 = 0.8,
		判断第二个符号: mid = low + range * P(MPS) = 0 + 0.8 * 0.8 = 0.64,由于 value < mid,所以符号为 1 (MPS),
		更新区间:high = mid = 0.64
		\item 计算范围: range = high - low = 0.64 - 0 = 0.64,
		判断第三个符号: mid = low + range * P(MPS) = 0 + 0.64 * 0.8 = 0.512,由于 value >= mid,所以符号为 0 (LPS),
		更新区间:low = mid = 0.512

		\item 计算范围: range = high - low = 0.64 - 0.512 = 0.128,
		判断第四个符号: mid = low + range * P(MPS) = 0.512 + 0.128 * 0.8 = 0.6144,由于 value >= mid,所以符号为 0 (LPS),
		更新区间:low = mid = 0.6144

		\item 计算范围: range = high - low = 0.64 - 0.6144 = 0.0256,
		判断第五个符号: mid = low + range * P(MPS) = 0.6144 + 0.0256 * 0.8 = 0.63488,由于 value < mid,所以符号为 1 (MPS),
		更新区间:high = mid = 0.63488

		\item 计算范围: range = high - low = 0.63488 - 0.6144 = 0.02048,
		判断第六个符号: mid = low + range * P(MPS) = 0.6144 + 0.02048 * 0.8 = 0.630784,由于 value >= mid,所以符号为 0 (LPS),
		更新区间:low = mid = 0.630784
	\end{serialNumber}

	通过上述步骤可以解码出原始序列 110010。
\end{tcolorbox}


\newpage
\section{CABAC编码}
CABAC(Context-Adaptive Binary Arithmetic Coding)是一种用于视频压缩的熵编码算法,广泛应用于H.264/AVC和HEVC(H.265)视频编码标准中。CABAC通过利用上下文信息来自适应地编码二进制符号,从而实现高效的压缩。其主要特点:
\begin{itemize}
	\item 上下文自适应:CABAC根据符号的上下文信息(如前面符号的统计特性)来选择概率模型,从而提高编码效率。
	\item 二进制算术编码:CABAC使用算术编码对二进制符号进行编码,这种方法比传统的霍夫曼编码更接近熵极限,能更好地压缩数据。
	\item 高压缩率:由于其自适应性和高效的算术编码,CABAC通常能比其他熵编码方法(如CAVLC)提供更高的压缩率。
\end{itemize}

CABAC编码过程主要包括以下几个步骤:
\begin{serialNumber}
	\item 上下文变量的初始化:确定所有上下文的初始MPS以及初始状态pStateIdx。
	\item 二进制化:将非二进制符号(如运动矢量、残差系数等)转换为二进制符号。
	\item 上下文建模:根据符号的上下文信息选择合适的概率模型。上下文信息可以包括前面符号的值、位置等。
	\item 算术编码:使用选定的概率模型对二进制符号进行算术编码。
\end{serialNumber}
详细来说,如图
\begin{figure}[H]
	\centering
	\includegraphics[width=0.8\textwidth]{1.png}
	\caption{}
	\label{fig:}
\end{figure}



\subsubsection{初始化上下文变量}
%\subsubsection{上下文建模}
%\begin{definition}
%	上下文建模是指在进行数据编码时,利用当前符号(或数据)的上下文信息来预测其概率分布,从而更有效地进行编码。
%\end{definition}
%
%上下文:可以是前面已经编码的符号或数据。例如,在编码一个文本时,当前字符的上下文可能是前面几个字符。
%
%概率估计:算术编码通过上下文建模来估计当前符号出现的概率。如果某个符号在特定上下文中出现的频率较高,编码器就会为这个符号分配一个较小的概率值,这样可以用更少的比特位表示它。
%
%动态更新:上下文模型可以是静态的(使用固定的概率表)或动态的(根据编码过程中遇到的数据实时更新概率)。动态模型通常能提供更好的压缩效果,因为它能够适应数据的变化。

CABAC 是基于上下文的概率编码,因此编码效率很大程度上依赖于上下文的准确性和初始化。
在每个Slice解码前,都需要初始化上下文变量,
通过根据当前片的类型和量化参数初始化上下文变量,编码器可以更好地估计符号的概率分布。
\begin{remark}
	亮度量化参数影响了,pStateIdx概率模型变量的大小范围构建。
\end{remark}

CABAC中使用了1024个上下文索引能够处理不同类型的数据元素和符号,这些索引提供了足够的细分以精确地模拟不同情况下的符号概率。视频编码过程中涉及多种类型的数据符号,如预测模式、变换系数的大小、运动向量的差分等。

\paragraph{m和n变量}
m 和 n 是根据Slice类型来选择的,不同的Slice类型(I、P、B)使用不同的上下文模型,用于根据Slice类型和量化参数对 CABAC 上下文进行初始化。它影响了上下文模型的初始概率状态,对于不同的Slice类型,能有更佳的初始概率。

\paragraph{valMPS概率模型}
\begin{definition}
	valMPS概率模型指示当前上下文模型中的“最可能符号”(Most Probable Symbol, MPS)。valMPS 的值对于确定在给定的上下文中哪个二进制值(0或1)具有较高的出现概率,它是一个二进制值(0或1),它标识了在某个特定的概率状态索引 pStateIdx 下,哪个符号(0或1)被认为是更可能发生的符号。
\end{definition}
\begin{remark}
	反之,LPS为最小概率符号。
\end{remark}

当一个符号被编码时,不仅 pStateIdx 会根据符号的实际值相对于 valMPS 的匹配情况被更新,valMPS 本身也可能会根据编码结果和当前的概率状态进行调整。例如,如果当前的 valMPS 是1,而编码的符号是0(即较不可能的符号,或LPS),并且根据 pStateIdx 的概率状态转换规则,概率状态可能会逆转,此时 valMPS 可能会从1变为0。

\paragraph{pStateIdx概率状态索引模型}
\begin{definition}
	pStateIdx(Probability State Index)是指在上下文建模中,用于表示当前上下文状态的索引。这个索引是决定二进制符号如何被编码的状态机的一部分,具体用于管理和更新概率估计。
\end{definition}

pStateIdx的工作原理:
\begin{serialNumber}
	\item pStateIdx: 它是一个范围从0到63的整数,表示当前上下文的概率状态。这个索引对应于编码一个二进制符号(0或1)的概率。每个索引值都有一个预定义的概率值,指示该状态下符号为“1”(MPS)的概率。
	\item 概率更新: 在CABAC中,每编码一个符号后,都会根据符号的实际值和当前的概率估计来更新pStateIdx。如果编码的符号与当前的MPS相同,则可能使pStateIdx向更稳定的状态(通常是更高的索引值,代表更确定的概率)移动;如果不同,则向不稳定的状态移动(通常是更低的索引值,代表更不确定的概率)。
	\item 概率状态的转换: CABAC定义了一套规则来调整pStateIdx,以便快速适应视频内容中符号概率的变化。这些规则确保编码过程中概率估计既快速又准确。
	\item MPS和LPS的角色: 最可能符号(MPS)和较少可能符号(LPS)是编码过程中用来更新状态的重要元素。pStateIdx的调整反映了符号值与这些预定义的“最可能”或“较少可能”的符号之间的关系。
\end{serialNumber}

\subsubsection{初始化上下文解码引擎}
\paragraph{codIRange(编码间隔)}
codIRange 表示当前编码的间隔(或概率间隔),它是一个表示概率范围的变量。在算术编码的开始,codIRange 通常被初始化为一个高值(如在 H.264 中初始化为 510),它代表了最大的概率区间。
在编码过程中,这个间隔会根据要编码的符号和其对应的概率模型被不断地细分。每次符号的编码都会根据符号的概率将当前的 codIRange 分割成两部分,反映了当前符号为0或1的概率。

在编码每个二进制符号时,根据该符号的概率模型(pStateIdx, valMPS),codIRange 被更新以反映下一个符号的概率。这一过程涉及到将 codIRange 乘以符号概率,然后根据编码的符号是MPS还是LPS来调整。


\paragraph{codIOffset(编码偏移)}
codIOffset 表示当前编码过程中的位偏移,它累积了已经编码的位流。这个偏移量在每次符号编码时根据 codIRange 的调整被更新。初始时,codIOffset 也被设置为一个低值(如0),随着编码过程的进行,这个值会增加,表示已经编码的数据量。

当 codIRange 更新后,codIOffset 可能需要进行范围调整,以保持编码过程的精度和避免溢出。当 codIRange 和 codIOffset 合起来超出了他们处理的位宽(如32位),就需要将部分已经确定的位输出到最终的编码位流中,并相应地更新 codIOffset。

codIRange 和 codIOffset 用于算术编码过程中管理编码间隔和位流的偏移。这两个参数共同协助在编码和解码过程中精确地处理概率和位流的动态更新。

\subsubsection{二值化}
%二值化过程的类型、maxBinIdxCtx 和 ctxIdxOffset。一元 (U) 二值化过程、截断一元 (TU) 二值化过程、级联一元/k 阶 Exp-Golomb (UEGk) 二值化过程和固定长度 (FL) 二值化过程的规范在分别第 9.3.2.1 至 9.3.2.4 条。其他二值化在第 9.3.2.5 至 9.3.2.7 节中指定。

二值化也由前缀和后缀位串的串联组成。对于这些二值化过程,前缀和后缀位串使用 binIdx 变量单独索引,这两组前缀比特串和后缀比特串分别称为二值化前缀部分和二值化后缀部分。

与语法元素的每个二值化或二值化部分相关联的是上下文索引偏移(ctxIdxOffset)变量的特定值和maxBinIdxCtx变量的特定值,


%如果没有为表 9-34 中标记的相应二值化或二值化部分的 ctxIdxOffset 分配值作为“na”,相应二值化或二值化前缀/后缀部分的比特串的所有bin通过调用第9.3.3.2.3节中指定的DecodeBypass过程来解码。在这种情况下,bypassFlag被设置为等于1,其中bypassFlag用于指示为了从比特流解析bin的值,应用DecodeBypass过程。

%ctxIdx = ctxIdxOffset = 276 被分配给语法元素 end_of_slice_flag 和 mb_type 的 bin,其指定 I_PCM 宏块类型,


\paragraph{ctxIdx的推导过程}
表9-11列出了每种切片类型需要初始化的ctxIdx。还列出了表号,其中包括初始化所需的 m 和 n 值。对于 P、SP 和 B 切片类型,初始化还取决于 cabac\_init\_idc 语法元素的值。请注意,语法元素名称不会影响初始化过程。

上下文索引 ctxIdx 的可能值在 0 到 1023 范围内(包含 0 和 1023)。分配给 ctxIdxOffset 的值指定分配给语法元素的相应二值化或二值化部分的 ctxIdx 范围的下限值。

ctxIdx,用于选择概率状态和MPS(Most Probable Symbol,最可能符号)


\paragraph{解码的三种模式}
Bypass (旁路解码)模式:解码不依赖于上下文模型的状态,而是简单地从比特流中直接读取下一个比特。

终止模式:用于解码数据流的结尾。解码终止符号。

决策模式:根据指定的上下文索引 ctxIdx 解码二进制符号。这个方法使用 CABAC 的概率模型来确定符号的值,并可能更新上下文模型以反映已观察到的数据的统计特性。



\newpage
\section{CABAC解码}
CABAC解码在H264一般是用于解码Slice Data的数据,具体来说流程图(见166 ITU-T Rec. H.264 (05/2003))


\newpage
\section{熵编码}
\begin{definition}
	信息熵是一个衡量信息不确定性的概念,通常用在信息论中。它由克劳德·香农(Claude Shannon)在1948年提出,用于量化信息的内容或复杂性。
	信息熵通常用符号 \( H(X) \) 表示,其中 \( X \) 是随机变量,其可能的结果有 \( x_1, x_2, ..., x_n \)。信息熵的计算公式如下:
	\begin{equation}
		H( X ) = - \sum_{i=1}^{n} p( x_i  ) \log_2 p( x_i )
	\end{equation}
\end{definition}
其中,\( p(x_i) \) 是事件 \( x_i \) 发生的概率。这样计算出的熵单位是“比特”(bit)。

假设有一个均匀分布的骰子,六个面都有相同的概率出现,每个面的概率 \( p(x_i) = \dfrac{1}{6} \),则熵计算为:
\begin{equation}
	H(X) = -\left(6 \times \frac{1}{6} \log_2 \frac{1}{6}\right) = -\log_2 \frac{1}{6} = 3 ~ bits
\end{equation}

\newpage
\section{代码实现}
\subsection{截断莱斯二元化}
\begin{lstlisting}[language=c++]
#include <bitset>
#include <iostream>

using namespace std;

// 编码函数
string riceEncode(int N, int k) {
  /* 0001 0111 >> 3 = 0010 = 2 */
  int q = N >> k;             // 商
                              /* 0001 0111 & 1000 - 1 = 111 = 3 */
  int r = N & ((1 << k) - 1); // 余数

  // 编码商
  string encoded = string(q, '1') + "0";

  // 编码余数
  bitset<32> binaryR(r);
  string binaryRStr = binaryR.to_string().substr(32 - k, k);
  encoded += binaryRStr;

  return encoded;
}

// 解码函数
int riceDecode(const string &encoded, int k) {
  int q = 0;
  int readToZeor = 0;

  // 解码商
  while (encoded[readToZeor] == '1') {
    q++;
    readToZeor++;
  }
  readToZeor++; // 跳过 '0'

  // 解码余数
  int r = 0;
  for (int i = 0; i < k; i++) {
    /* 这里减去字符'0'是字符和整数之间的转换 */
    r = (r << 1) | (encoded[readToZeor] - '0');
    readToZeor++;
  }

  return (q << k) + r;
}

// 比如,对23进行编码,那么选定任意$k = 3$,则$23 = q * 2^3 + r$,得出q = 2 , r = 7,则q = "110" , r = "111",则编码结果为"110111"
// 反之,对"110111"进行解码,那么先读到第一个'0',即"110",解码为 q = 2, 根据参数k=3可知,再继续读取3位则为余数r = b'111' = 7, 故$2 * 2^3 + 7 = 23$

const int k = 3; // 参数 k
int main() {
  int N = 23; // 要编码的数
  cout << "Encoded: " << riceEncode(N, k) << endl;
  cout << "Decoded: " << riceDecode("110111", k) << endl;
  return 0;
}
\end{lstlisting}

%--------------------------------- Reference --------------------------------
\newpage
\begin{thebibliography}{1}
	\bibitem{a} 作者. \emph{文献}[M]. 地点:出版社,年份.\url{www.google.com}
\end{thebibliography}

\end{document}
